\begin{normalsize}

\subsection{Теорема Ван-дер-Вардена}

\begin{theorem} [Ван Дер Вардена, 1927]
    Пусть натуральный ряд раскрашен в конечное число цветов. Тогда в нем можно найти сколь угодно длинную конечную одноцветную арифметическую прогрессию.

    Более формально:

    Для любых $k, c \in \N$ существует $W \in \N$, такое что для любой раскраски $\chi: \{1, \ldots, W\} \to \{1, \ldots, c\}$ существуют $a,d \in \N$, такие что 
    \[ \chi(a) = \chi(a + d) = \chi(a + 2d) = \ldots = \chi(a + (k - 1)d). \]
\end{theorem}

\begin{defn}
    Числом Ван дер Вардена $W(k, c)$ называется наименьшее число $W$ удовлетворяющее теореме Ван дер Вардена с параметрами $(k, c)$.
\end{defn}

На самом деле, утверждение верно для всякого цвета положительной плотности:

\begin{theorem} [Семереди]
    Для любой плотности $\delta \in (0, 1)$ и для любого $k \in \N$ имеетсся число $N(k, \delta)$ такое, что любое подмножество $A \subseteq \{1, \ldots, N\}$ мощности $\delta N$ содержит арифметическую прогрессию длины $k$ для любого $N > N(k, \delta)$
\end{theorem}

\begin{proof}
    Без доказательства.
\end{proof}

\end{normalsize}
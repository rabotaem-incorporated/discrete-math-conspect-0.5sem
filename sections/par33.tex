\subsection{Теорема Брукса}

\begin{theorem}[Брукс, 1941]
    Пусть в графе $G$ степени всех вершин $\leq d$. Тогда если $d \geq 3$ и ни одна компонента связности $G$ не является полным графом $K_{d+1}$, то $\chi(G) \leq d$.

    При $d = 2$ неравенство $\chi(G) \leq 2$ выполняется, если ни одна компонента связности не является нечетным циклом.
\end{theorem}

\begin{statement}
    Пусть $u, v$ --- две несмежные вершины графа $H$.

    Рассмотрим графы:

    $H / uv$: вершины $u, v$ склеены в одну(кратные ребра после склеивания сразу делаются простыми),

    $H + uv$: добавлено ребро $uv$.

    Тогда граф $H$ можно покрасить в $k$ цветов $\iff$ хотя бы один из этих двух графов можно.
\end{statement}

\begin{proof}[ утверждения]
    Раскраскам $H$, в которых вершины $u$ и $v$ одного цвета, соответствуют покраски $H / uv$, а тем, в которых $u$ и $v$ разного цвета, покраски $H + uv$.
\end{proof}

\begin{proof}[ теоремы Брукса]

    Для $d = 2$ утверждение следует из того, что четные циклы и пути легко красятся в два цвета, а других компонент связности в нашем графе нет.

    Пусть $d \geq 3$. Можно считать граф $G$ связным.

    Преположим противное: пусть для графов с меньшим числом вершин, чем у $G$, утверждение теоремы Брукса выполняется, а для графа $G$ нет.

    Лемма \ref*{lem:geqk}    $\implies$ степени всех вершин графа $G$ равны $d$ (иначе возьмем подграф с таким свойством, связность $\implies$ совпадает с $G$).

    Рассмотрим любую вершину $p$ графа $G$, у нее найдутся два несмежных соседа $u, v$ (иначе $G$ совпадает с $K_{d+1}$).

    Рассмотрим граф $G / uv$ ($z$ --- вершина, получаемая отождествлением $u$ и $v$).
    \begin{itemize}
        \item его не покрасить в $d$ цветов в силу утверждения
        \item он связен
        \item степени всех его вершин $\leq d$, кроме, возможно, $z$.
        \item $\deg{p} < d$.
    \end{itemize}

    По Лемме \ref*{lem:geqk} в $G / uv$ найдется индуцированный подграф $H$, в котором степени всех вершин $\geq d$.

    $H$ получается из некоторого подграфа $H'$ графа $G$ стягиванием ребра $uv$.

    $H$ содержит $z$, так как степени остальных и так не больше $d$, так что при удалении $z$ из-за связности мы должны были удалить какие-то ведущие в них ребра.

    $\implies$ в $H$ есть вершина $z$ и несколько вершин степени $d$, которые не смежны с вершинами вне $H$.

    Попробуем теперь покрасить граф $G + uv$.
    \begin{itemize}
        \item граф $\tilde{H}$: состоит из $u, v$ и вершин, не входящих в $H'$
        
        \item $G + uv$ состоит из $H'$ и $\tilde{H}$; эти два графа имеют общее ребро $uv$, а их вершины, отличные от $u$ и $v$, не смежны.
        
        \item $\implies$ если мы покрасим каждый из них в $d$ цветов, то, склеивая эти раскраски по ребру $uv$, получим раскраску $G + uv$.
        
        \item Покажем что степени всех вершин графов $H', \tilde{H}$ не превосходят $d$.
        \begin{itemize}
            \item для всех вершин, кроме $u$ и $v$, это очевидно, и проверить следует лишь что, например, вершина $u$, имеющая степень $d+1$ в графе $G + uv$, соединена не только с вершинами $H'$ или не только с вершинами $\tilde{H}$
        
            \item Для $H'$: вершина $u$ соединена с вершиной $p$, лежащей в $\tilde{H}$.
        
            \item Для $\tilde{H}$: если $u$ соединена только с вершинами $\tilde{H}$, то в $H$ имеем $\deg{z} < d$ (ребрам, выходящим в $H$ из $z$, будут соответствовать лишь ребра, выходящие в $G$ из $v$, причем не все --- например, не ребро $zp$).
        \end{itemize}
    \end{itemize}
    Таким образом, для графов $H', \tilde{H}$ (оба имеют меньше вершин, чем $G$) выполняется утверждение теоремы Брукса, так что один из них $K_{d+1}$.

    Это не $H'$, так как в $H'$ вершины $u$ и $v$ не имеют общих соседей (такой общий сосед имел бы степень меньше $d$ в $H$ и потому попал бы в $\tilde{H}$).

    И это не $\tilde{H}$, поскольку в этом случае степень $z$ в $H'$ получается не больше двух --- противоречие.
\end{proof}
\subsection{Паросочетания с предпочтениями}

\begin{defn}
    У каждой вершины можно задать порядок на множестве инцидентных ей ребер: $<_v \subseteq E \times E$ (предпочтения).
\end{defn}

\begin{defn}
    Паросочетание $M$ называется устойчивым, если не существует $(v_1, v_2) \in E \setminus M$, которое удовлетворяет следующим условиям:
    
    \begin{itemize}
        \item ребро $(v_1, v_2)$ у $v_1$ стоит выше в списке предпочтений, чем его текущая пара $(v_1, v_2') \in M$ (либо $v_1$ не состоит в паре);
        
        \item симметричное условие для $v_2$: ребро $(v_1, v_2)$ у него стоит выше в списке предпочтений, чем его текущая пара $(v_1', v_2) \in M$ (либо $v_2$ не состоит в паре)
    \end{itemize}
\end{defn}

\begin{examples}~
    \begin{itemize}
        \item $n$ мужчин, $n$ женщин, полный порядок $(K_{n,n})$
       
        \item ориентированные ребра
    \end{itemize}
\end{examples}

\begin{theorem}[об устойчивых браках, Гейл и Шепли, 1962]
    Во всяком двудольном графе $G = (V_1, V_2, E)$, для всяких предпочтений $\{\leq_v\}_{v \in V_1 \cup V_2}$ существует устойчивое паросочетание.
\end{theorem}

\begin{proof}
    Алгоритм, строящий такое паросочетание ($V_1$ --- юноши, $V_2$ --- невесты).

    \textbf{Описание алгоритма:}

    Первый шаг:
    \begin{itemize}
        \item каждый юноша делает предложение первой девушке в своем списке
        
        \item каждая девушка заключает помолвку с наиболее предпочтительным женихом сделавших ей предложение
    \end{itemize}
    
    Каждый следующий шаг:
    \begin{itemize}
        \item каждый не помолвленый юноша делает предложение следующей девушке в своем списке --- неважно, помолвлена она или нет.
        
        \item если девушка получает предложение от более предпочтительного жениха, чем ее текущий жених, то она расторгает текущую помолвку с наиболее предпочтительным женихом их тех, кто сделал ей предложение.
    \end{itemize}
    
    Постепенно заключаются помолвки, все более предпочтительные для невест, и все менее предпочтительные для женихов. Ни одна юноша не делает предложения одной и той же девушке дважды.

    \textbf{Корректность алгоритма:}

    Алгоритм завершается, поскольку на каждом шаге хотя бы один юноша последовательно движется по своему списку предпочтений, общее число шагов ограничено сверху суммой длин этих списков.
    
    \textbf{Устойчивость} полученного паросочетания $M$:

    Для всякой несложившейся пары $(v_1, v_2) \in E \setminus M$ рассмотрим следующие случаи.
    \begin{itemize}
        \item $v_1$ никогда не делал предложение $v_2 \implies$ к моменту завершения алгоритма у него была более предпочтительная невеста, чем $v_2$, и, женившись на ней, менять ее на $v_2$ он не захочет. То есть существует $v_2'$, такая что $(v_1, v_2') \in M$ и это ребро выше в предпочтении $v_1$, чем $(v_1,v_2)$.
        
        \item $v_1$ делал предложение $v_2$, но получил отказ $\implies$ к этому моменту у $v_2$ был более предпочтительный жених, которого она могла сменить только на еще более предпочтительного. То есть $\exists v_1'$, такой что $(v_1', v_2) \in M$ и это ребро выше в предпочтении $v_2$, чем $(v_1,v_2)$.
        
        \item $v_1$ делал предложение $v_2$, получил согласие, а потом был брошен ею $\implies$ у $v_2$ есть более предпочтительный жених, то есть $\exists v_1'$, такой что $(v_1', v_2) \in M$ и это ребро выше в предпочтении $v_2$, чем $(v_1,v_2)$.
    \end{itemize}
\end{proof}

\subsubsection*{Свойства полученного устойчивого паросочетания}

\begin{itemize}
    \item для $K_{n,n}$ образуется $n$ пар
    
    \item оптимально для мужчин (то есть каждый мужчина женат на наиболее предпочтительной им женщине среди всех устойчивых паросочетаний)
    
    \item самое худшее для женщин (то есть каждая женщина замужем за наименее предпочтительным мужчиной среди всех устойчивых паросочетаний)
\end{itemize}

\begin{proof}[ наличия $n$ пар]
    
    Пусть получилось меньше $n$ пар, тогда существует мужчина который сделал предложение всем женщинам, и женщина не получившая ни одного предложения. Противоречие.
\end{proof}

\begin{notice}
    Возможная пара $(m, w)$: $\exists$ стабильное паросочетание с такой парой.
\end{notice}

\begin{notice}
    Наилучший возможный партнер $w = best(m)$ для $m$: наиболее предпочтительный среди возможных пар $(m, w)$.
\end{notice}

\begin{proof}[ оптимальности для мужчин]

    Предположим, что в паросочетании GS, выданным алгоритмом, есть мужчина, который не с наилучшей возможной партнершей. Значит, его наилучшая возможная партнерша ему отказала.

    Рассмотрим первое событие $X$, когда мужчине отказала наилучшая возможная партнерша во время работы GS:

    $w = best(m)$ отказала $m$, чтобы быть (или продолжать быть) с мужчиной $m'$, более предпочтительным, чем $m$.

    Так как $(m, w)$ возможная пара, то $\exists$ стабильное паросочетание $S'$ с такой парой.

    Обозначим партнершу $m'$ в $S'$ за $w' \neq w$. Пара $(m', w')$ --- возможная.

    Алгоритм GS $\implies$ во время события $X$
    \begin{itemize}
        \item $m'$ еще не был отвергнут $best(m') \implies$ и никем из возможных партнерш, в том числе $w'$
       
        \item $m'$ состоит в паре с $w$, то есть мужчине $m'$ отказали все женщины в его списке предпочтений выше $w$
    \end{itemize}
    
    $\implies w'$ после $w$ в списке предпочтений $m'$.
    
    Противоречие со стабильностью $S'$: $(m, w),~(m', w') \in S'$, но оба $w$ и $m'$ предпочитают друг друга относительно их пар в $S'$. Оптимальность для мужчин доказана.
\end{proof}

\begin{proof} [ <<наихудшести>> для женщин]

    Пусть $S$ --- паросочетание найденное алгоритмом Гейла-Шепли. Пусть $(m, w)$ --- стабильная пара в $S$, но $m$ не самый худший выбор для $w$ среди всех стабильных паросочетаний. 
    
    Тогда существует паросочетание $S'$, в котором есть пара $(m', w)$, где $m'$ хуже $m$ в списке $w$.

    Тогда существует пара $(m, w')$. Но по оптимальности для мужчин $w$ оптимальнее $w'$. Пара $(m, w)$ неустойчивая, значит $S'$ --- нестабильное паросочетание. Противоречие.
    
\end{proof}
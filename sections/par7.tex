\subsection{Многочлен Жегалкина}

\begin{defn}
    Многочлен Жегалкина --- сумма по модулю два конъюнкций переменных (также допускается слагаемое-единица) без повторения слагаемых, а также константа 0.\\
    \example $f(x, y, z) = 1 \oplus x \oplus (x \wedge y \wedge z)$
\end{defn}

\subsubsection*{Общий вид}

\[ \def\useanchorwidth{F}\stackMath
    f(x_1, \ldots, x_n) = a \oplus \mathop{\stackunder{\bigoplus}{\def\stackalignment{l}%
    \stackunder[2pt]{\scriptscriptstyle 1 \leq i_1 < \ldots < i_k \leq n}
                    {\scriptscriptstyle k \in \{1, \ldots, n\}}
    }} a_{i_1 \ldots i_k} \wedge x_{i_1} \wedge \ldots \wedge x_{i_k}\]
где $a_i, a_{i_1 \ldots i_k} \in \{0, 1\}$\\

\begin{notice}
    Зачастую константу 0 не считают полиномом Жегалкина, то есть в выражении допускаются только конъюнкции, сложения и константа 1.
\end{notice}

\begin{theorem}
    Для каждой функции существует единственное представление многочленом Жегалкина.
\end{theorem}

\begin{proof}

    \textsl{Существование:} Преобразуем ДНФ:

    замена дизъюнкции: $x \vee y = x \oplus y \oplus (x \wedge y)$
    
    замена отрицаний: $\neg x = x \oplus 1$ 
    
    раскрытие скобок: $(x \oplus y) \wedge z = (x \wedge z) \oplus (y \wedge z)$
    
    сокращение одинаковых слагаемых: $x \oplus x = 0$
    
    \textsl{Единственность:} всего многочленов Жегалкина $2^{2^{n}}$; функций столько же $\implies$ представление единственно.
\end{proof}
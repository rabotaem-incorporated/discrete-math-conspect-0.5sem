\subsection{Гамильтоновы циклы и пути}

\begin{defn}
    Простой путь или цикл в графе называется гамильтоновым, если он проходит через каждую вершину ровно один раз.
\end{defn}

В отличии от эйлерового пути, простых критериев существования гамильтонова пути или цикла в графе неизвестно. (NP-полная задача --- позже в курсе; 1000000\$ за решение от института Клея).

\begin{lemma}
    Если в графе с $k \geq 3$ вершинами имеется гамильтонов путь, и сумма степеней концов этого пути не меньше, чем $k$, то в нем имеется и гамильтонов цикл.
\end{lemma}

\begin{proof}
    
    Пусть $p = A_1 A_2 \ldots A_k$ гамильтонов путь, и вершина $A_1$ имеет степень $l$.

    Назовем зелеными вершины, предшествующие (в смысле порядка от $A_1$ до $A_k$) в пути $p$ тем $l$ вершинам, с которыми смежна $A_1$. Очевидно, зеленых вершин ровно $l$.

    Предположим, что вершина $A_k$ не соединена с зелеными вершинами. Тогда степень вершины $A_k$ не больше $k - 1 - l$, то есть сумма степеней вершин $A_1$ и $A_k$ не больше $k - 1$ --- противоречие.

    Значит, вершина $A_k$ соединена с какой-то зеленой вершиной $A_i$. В этом случае в графе существует гамильтонов цикл $A_1 A_2 \ldots A_i A_k A_{k - 1} \ldots A_{i + 1} A_1$.
\end{proof}

Достаточное условие существования гамильтонова пути или цикла в терминах степеней вершин.

\begin{theorem}[Дирак, 1952]~
    
    Если в графе $G$ с $n \geq 3$ вершинами сумма степеней любых двух вершин не меньше $n - 1$ (соответственно, не меньше $n$), в нем существует гамильтонов путь (соответственно цикл).

\end{theorem}

\begin{proof}
    
    Лемма $\implies$ если теорема верна для пути, то верна и для цикла. Докажем для пути.

    Рассмотрим самый длинный простой путь $p$. Предположим, что он не гамильтонов, и содержит $k < n$ вершин.

    Граф, образованный вершинами пути $p$ назовем $H$.

    Концы самого длинного пути $p$ соединены только с другими вершинами $p$, так что к $H$ применима лемма: сумма степеней концов пути $p$, являющегося в $H$ гамильтоновым, не меньше чем $n - 1 \geq k$ (легко видеть, что $k \geq 3$, так что лемму применять можно).

    Следовательно, в $G$ есть цикл длины $k$.

    Если из него ведет хотя бы одно ребро вне цикла, то имеем путь длины $k + 1$. Противоречие с максимальностью $p$.

    Иначе степени всех вершин цикла $\leq k - 1$, а степени не входящих в цикл вершин $\leq n - k - 1$, что в сумме дает $\leq n - 2$. Противоречие.

\end{proof}
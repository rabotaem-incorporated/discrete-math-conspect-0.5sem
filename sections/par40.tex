\subsection{Вершинное покрытие}

\begin{defn}
    Вершинное покрытие графа --- это такое множество вершин, что каждое ребро содержит хотя бы одну из них.
\end{defn}

\begin{theorem}[Кениг, 1931]
    Наибольшее число ребер в паросочетании двудольного графа $G$ равно наименьшему числу вершин в вершинном покрытии графа $G$.
\end{theorem}

\begin{proof}
    Применим теорему Геринга к графу $G$ и множествам, состоящим из вершин одной и второй доли. Заметим, что каждый путь можно сократить только до одного ребра, так что наибольшее количество путей есть просто наибольшее паросочетание, а разделяющее множество --- это в точности вершинное покрытие.
\end{proof}
\subsection{Раскраски графов}

\begin{defn}
    $G = (V, E), C$ --- множество цветов.

    Раскраска --- это всякая функция $c: V \to C$.
\end{defn}

\begin{defn}
    Раскраска правильная, если для всякого ребра $(v, u)$ верно $c(v) \neq c(u)$.
\end{defn}

\begin{examples}~
    \begin{itemize}
        \item Двудольный граф: если раскрашивается в два цвета.
        \item Граф $K_n$ не раскрашивается менее чем в $n$ цветов.
    \end{itemize}
\end{examples}

\begin{theorem}[Хивуд]
    Всякий планарный граф раскрашивается в $5$ цветов.
\end{theorem}

\begin{proof}

    Индукция по числу вершин.

    Базис: $|V| = 1$: раскрашивается в один цвет.

    Шаг индукции: По следствию из теоремы \ref*{thm:3v6e}, есть вершина $v$ степени $\leq 5$.

    Если $\deg{v} \leq 4$, то удаляем ее, остаток раскрашиваем по предположению индукции, а затем возвращаем и раскрашиваем в свободный цвет.

    Если $\deg{v} = 5$, то рассмотрим ее соседей $v_0, v_1, v_2, v_3, v_4$ в порядке их укладки на плоскости. Если какого-то из ребер $(v_i, v_{i+1})$ нет, добавим его.
    
    Хотя бы одной из диагоналей $(v_i, v_{i+2})$ нет, иначе был бы подграф $K_5$.

    Склеим вершины $v_i, v_{i+2}$ и $v$ --- получим планарный граф меньшего размера, который раскрашивается в 5 цветов по предположению индукции. Тогда в исходном графе $v_i$ и $v_{i+2}$ покрасим в тот же цвет, что и склеенную вершину, а $v$ --- в свободный пятый цвет.
\end{proof}

\begin{theorem}[Аппель, Хакен, 1977]
    Всякий планарный граф раскрашивается в $4$ цвета.
\end{theorem}

\begin{proof}
    Компьютерный перебор(первое в истории доказательство такого рода).
\end{proof}
\subsection{Критерий полноты системы функций}

\begin{defn}
    Полная система --- множество булевых функций $\mathcal{F}$, где все булевы функции выразимы формулами над этим базисом.
\end{defn}

\begin{theorem}[Пост, 1921]
    Множество булевых функций $\mathcal{F}$ является полным тогда и только тогда, когда $\mathcal{F}$ не содержится ни в одном из пяти классов $T_0,\ T_1,\ S,\ M,\ L$.
\end{theorem}

\begin{proof}

    "$\Rightarrow$":
    
    Если содержится, то его замыкание $[\mathcal{F}]$ также содержится в этом классе. А значит система не полная.
    
    "$\Leftarrow$":
    
    Пусть не содержится, то есть есть функции $f_0, f_1, f_S, f_M, f_L \in \mathcal{F}$, где $f_0 \notin T_0, f_1 \notin T_1, f_S \notin S, f_M \notin M, f_L \notin L$.(Эти функции не обязательно различны)

    \textsl{План доказательства:}
    \begin{enumerate}
        \item Из $f_0, f_1$ выражается либо отрицание, или обе константы, или и то и другое.
        \item Выразим отрицание и константы следующим образом:
        \begin{enumerate}
            \item[2.1] Если получилось отрицание, то из $f_S$ выражаются константы.
            \item[2.2] Если же вышли обе констаты, то отрицание выражается из $f_M$.
        \end{enumerate}
        \item Из $f_L$ выражается конъюнкция.
    \end{enumerate}
    
    Получив константу, отрицание и конъюнкцию можно выразить любую булеву функцию(например через ДНФ).

    (1) Так как $f_0 \notin T_0$, то по определению $T_0$ имеем $f_0(0, \ldots, 0) = 1$.
    \begin{enumerate}
        \item Если при этом $f_0(1, \ldots, 1) = 1$, то получена константа $1$ в виде $\varphi_1(x) = f_0(x, \ldots, x) = 1$.
        \item Если же $f_0(1, \ldots, 1) = 0$, то в таком виде получено отрицание, $\overline{\varphi}(x) = \neg x = f_0(x, \ldots, x)$. 
    \end{enumerate}
    Аналогично, для $f_1 \notin T_1$: известно, что $f_1(1, \ldots, 1) = 0$, и рассматривая значение $f_1(0, \ldots, 0)$, получаем или константу 0, или отрицание.
    
    (2.1) Пусть получено отрицание 
    Для функции $f_S \notin S$ известно, что существует набор $(\sigma_1, \ldots, \sigma_n)$, на котором

    \[f_S(\sigma_1, \ldots, \sigma_n) \neq \neg f_S(\neg\sigma_1, \ldots, \neg\sigma_n),\]
    то есть
    \[f_S(\sigma_1, \ldots, \sigma_n) = f_S(\neg\sigma_1, \ldots, \neg\sigma_n).\]
    Тогда формула $f_S(x^{\sigma_1}, \ldots, x^{\sigma_n})$, построенная из $f_S$ и из отрицания, выражает одну из констант. С помощью отрицания выражается вторая константа.

    (2.2) Пусть получены обе константы.

    Для функции $f_M \notin M$ существует два набора $\alpha$ и $\beta$, для которых $\alpha < \beta$, но $f_M(\alpha) = 1$ и $f_M(\beta) = 0$.

    Пусть $i_1, \ldots, i_k$ --- номера всех координат, в которых $\alpha$ и $\beta$ отличаются друг от друга. Соответственно, в $\alpha$ там 0, в $\beta$ --- 1, а остальные коордиаты общие, $\sigma_i$, где $i \notin \{i_1, \ldots, i_k\}$:

    \[f_M(\sigma_1, \ldots, \sigma_{i_1-1}, 0, \sigma_{i_1+1}, \ldots, \sigma_{i_k-1}, 0, \sigma_{i_k+1} \ldots \sigma_n) = 1,\]

    \[f_M(\sigma_1, \ldots, \sigma_{i_1-1}, 1, \sigma_{i_1+1}, \ldots, \sigma_{i_k-1}, 1, \sigma_{i_k+1} \ldots \sigma_n) = 0.\]
    
    Чтобы получить отрицание, подставим:
    
    \begin{itemize}
        \item костанты вместо всех общих координат
        \item одну и ту же переменную $x$ во все изменяющиеся координаты:
    \end{itemize}
    
    \[ \neg x = f_M(\sigma_1, \ldots, \sigma_{i_1-1}, x, \sigma_{i_1+1}, \ldots, \sigma_{i_k-1}, x, \sigma_{i_k+1} \ldots \sigma_n). \]

    (3) Так как функция $f_L$ нелинейна, ее многочлен Жегалкина содержит хотя бы одну конъюнкцию.

    Пусть переменные $x$ и $y$ входят в состав этой конъюнкции.

    Тогда функцию можно представить в виде $f_L(x, y, z, \ldots) = xyP(z, \ldots) \oplus xQ(z, \ldots) \oplus yR(z, \ldots) \oplus S(z, \ldots)$, где $P, Q, R, S$ --- многочлены Жегалкина($Q, R, S$ могут отсутствовать).

    Так как $P$ --- не константа 0, она равна единице на некотором наборе $\alpha$.

    Тогда $g(x, y) = f_L(x, y, \alpha) = xyP(\alpha) \oplus xQ(\alpha) \oplus yR(\alpha) \oplus S(\alpha) = xy \oplus xb \oplus yc \oplus d$, где $b, c, d \in \{0, 1\}$.

    Подстановкой $g(x \oplus c, y \oplus b)$ получается следующая функция:

    \[ h(x, y) = g(x \oplus c, y \oplus b) = (x \oplus c)(y \oplus b) \oplus (x \oplus c)b \oplus (y \oplus b)c \oplus d = xy \oplus bc \oplus d. \]

    В зависимости от значения константного слагаемого $bc \oplus d$, получилась или конъюнкция или ее отрицание. В последнем случае можно применить к ней ранее выраженную операцию отрицания.
\end{proof}
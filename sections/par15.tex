\subsection{Факториал}

\begin{lemma} (грубые оценки для $n!$)
    \[ \left( \frac{n}{e} \right)^n < n! < n^n \]
\end{lemma}

\begin{proof}
    Верхняя оценка очевидна.\\
    Докажем нижнюю оценку по индукции.

    \textsl{База:} $\left( \frac{1}{e} < 1 \right)$, верно.

    \textsl{Переход:} Пусть верно для $n$: $\left( \frac{n}{e} \right)^n < n!$.\\
    Покажем для $(n + 1)$.\\
    $(n + 1)! = (n + 1)n! > (n + 1) \left( \frac{n}{e} \right)^n$ (по предположению индукции).\\
    Надо показать: $(n + 1) \left( \frac{n}{e} \right)^n > \left( \frac{(n + 1)}{e} \right)^{n + 1}$.\\
    Это верно тогда и только тогда, когда $en^n > (n + 1)^n$.\\
    что в свою очередь равносильно $\left( 1 + \frac{1}{n} \right)^n < e$, что верно.
\end{proof}

\begin{theorem} (формула Стирлинга)
    \[ n! = (1 + o(1)) \sqrt{2 \pi n} \left( \frac{n}{e} \right)^n \]
\end{theorem}

\begin{proof}
    в курсе матанализа.
\end{proof}
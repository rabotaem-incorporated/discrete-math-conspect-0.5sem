\subsection{Мосты}

\begin{defn}
    Мост --- ребро, удаление которого увеличивает число компонент связности.
\end{defn}

\begin{theorem}[о мостах]
    
    Ребро является мостом тогда и только тогда, когда оно не принадлежит ни одному циклу.
\end{theorem}

\begin{proof}
    
    Необходимость:
    
    Предположим, что оно принадлежит некоторому циклу $e, e_1, \ldots, e_k$. Рассмотрим две произвольные вершины $u, v$ из одной компоненты связности в $G$, то есть они соединены некоторым путем в $G$. Если $e$ не принадлежит этому пути, то они им соединены и в $G \setminus e$.
    Если $e$ принадлежит этому пути, то заменив в нем ребро $e$ на последовательность ребер $e_1, \ldots, e_k$, получим что они соединены путем в $G \setminus e$.
    Следовательно, после удаления $e$ компонены связности не меняются, то есть $e$ не является мостом по определению.

    Достаточность:

    Пусть ребро $e = (x, y)$ не содержится ни в одном из циклов графа $G$.

    Вершины $x$ и $y$ связаны, то есть лежат в одной компоненте связности $G_1$ графа $G$.

    Если в графе $G \setminus e$ вершины $x$ и $y$ соединены путем, то прибавив к нему $e$, получим цикл, что невозможно по условию.

    Следовательно вершины $x$ и $y$ находятся в разных компонентах связности графа $G \setminus e$.

    Таким образом, после удаления ребра $e$ из $G$ компонента $G_1$ распалась как минимум на две компоненты связности, то есть число компонент связности увеличилось и $e$ --- мост по определению.
\end{proof}
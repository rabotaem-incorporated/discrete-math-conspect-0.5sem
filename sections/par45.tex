\begin{normalsize}

\subsection{Следствия}

\subsubsection*{В теории чисел}

\begin{theorem} [Шур, 1917]

    Если натуральный ряд покрашен в конечное число цветов, то уравнение $x + y = z$ имеет одноцветое решение.

\end{theorem}

\begin{proof}
    Рассмотрим полный граф, вершины которого натуральные числа, и покрасим ребро $(i, j)$ в тот цвет, в который покрашено число $(j - i)$. По теореме Рамсея в этом графе найдется одноцветный треугольник, то есть три числа $a < b < c$ такие, что числа $x = b - a$, $y = c - b$, $z = c - a$ одного цвета. Что и требовалось (достаточно даже ограниченного --- но зависящего от числа цветов --- отрезка натурального ряда).
\end{proof}

\subsubsection*{Обобщения}

То же верно для сумм не только двух, но и любого конечного числа $k$ элементов:

\begin{theorem}[Фолькман-Радо-Сандерс]
    
    Если натуральный ряд покрашен в конечное число цветов, то существует цвет $i$ и $k$ чисел $x_1, x_2, \ldots, x_k$, такие что $\forall I \subseteq \{1, \ldots, k\},~\sum\limits_{j \in I} x_j$ покрашены в цвет $i$.
    
\end{theorem}

Более того:

\begin{theorem} [Hindman, 1974]
    
    Если натуральный ряд покрашен в конечное число цветов, то существует цвет $i$ и бесконечная возрастающая последовательность чисел $x_1 < x_2 < \ldots$, такая что $\forall I \subseteq \N,~\sum\limits_{j \in I} x_j$ покрашены в цвет $i$.

\end{theorem}


\subsubsection*{В геометрии}

\begin{theorem} [Эрдеш, Секереш, 1935]

    Для любого натурального $k$ найдется такое $N$, что из любых $N$ точек на плоскости общего положения (никакие $3$ не лежат на одной прямой) найдется $k$, являющихся вершинами выпуклого $k$-угольника.
    
\end{theorem}

\begin{proof}

    Точки в выпуклом положении: вершины выпуклого многоугольника:

    иначе --- в невыпуклом положении.

    Два утверждения:

    \begin{itemize}
        \item Из любых пяти точек общего положения найдутся $4$ в выпуклом положении.
        
        \begin{proof}
        
            Если выпуклая оболочка нашей пятерки точек --- это четырехугольник или пятиугольник, то все ясно, если же это треугольник $ABC$ с точками $D, E$ внутри, то прямая $DE$ пересекает какие-то две стороны треугольника $ABC$, например, $AB$ и $AC$, и тогда $B, C, D, E$ в выпуклом положении.
        
        \end{proof}

        \item Если из $k \geq 4$ точек любые $4$ лежат в выпуклом положении, то все лежат в выпуклом положении.
        
        \begin{proof}

            в противном случае какая-то точка $P$ попадет внутрь выпуклой оболочки остальных. Если $A$ --- вершина выпуклой обоолочки, то луч $AP$ повторно пересечет некоторую сторону $BC$ выпуклой оболочки, и тогда $A, B, C, P$ не будут в выпуклом положении.
        
        \end{proof}

    \end{itemize}

    Покажем, что можно взять $N = R(4; 5, k)$.

    В самом деле, крася четверку точек в первый цвет, если она в невыпуклом положении, и во второй --- если в выпуклом, мы найдем либо $5$ точек, для которых все четверки первого цвета (что невозможно по $1.$), либо $k$ точек таких, что все четверки второго цвета, что нам и нужно по $2.$).
    
\end{proof}

\end{normalsize}
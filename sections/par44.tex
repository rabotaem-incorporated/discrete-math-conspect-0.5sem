\subsection{Оценки на числа Рамсея}

\begin{theorem} (Верхняя оценка чисел Рамсея)
    \[ R(n, m) \leq C_{n + m - 2}^{m - 1} \]
\end{theorem}

\begin{proof}

    Из доказательства теоремы Рамсея имеем $R(k; m_1, \ldots, m_d) \leq R(k-1; Q_1, \ldots, Q_d)$
    
    При $k = 2, d = 2$ получаем $R(2, m, n) \leq 1 + R(1; R(2, m - 1, n), R(2; m, n - 1)) \overset{\text{база 1}}{=}$

    $1 + R(2; m - 1, n) + R(2; m, n - 1) - 2 + 1$.

    Таким образом имеемм $R(n, m) \leq R(n - 1, m) + R(n, m - 1)$.

    По второй базе $R(1, m) = 1$. По индукции получаем нижнюю оценку:\\
    $R(n, m) \leq R(n - 1, m) + R(n, m - 1) \leq C_{n + m - 3}^{m - 1} + C_{n + m - 3}^{m - 2} = C_{n + m - 2}^{m - 1}$.\\
\end{proof}

\begin{follow} (Верхняя оценка на диагональные числа Рамсея)
    \[ R(n, n) \leq (1 + o(1)) \frac{4^{n - 1}}{\sqrt{\pi n}} \]
\end{follow}

\begin{proof}
    оценка центрального биномиального коэффицента из лекции про числа Каталана.
\end{proof}

\begin{theorem} (Нижняя оценка диагональных чисел Рамсея)
    \[ R(n, n) \geq 2^{\frac{n}{2}}~\text{при}~n \geq 2 \]
\end{theorem}

\begin{proof}
    Случай $n = 2$ понятен, так что пусть $n \geq 3$.

    Пусть $N < 2^{\frac{n}{2}}$. Всего графов на $N$ вершинах $2^{\frac{N(N-1)}{2}}$.

    Идея: если всего графов, содержащих клику, $< \frac{2^{\frac{N(N-1)}{2}}}{2}$ (аналогично для антиклики), то существует граф, в котором нет ни того, ни другого, то есть $N < R(n, n)$.

    Всего графов на $N$ вершинах, содержащих клику на данных $n$ вершинах, $2^{\frac{N(N-1)}{2} - \frac{n(n-1)}{2}}$. Всего $n$-клик $C_N^n$.

    Всего графов размера $N$, содержащих $n$-клику,

    \[ \leq C_N^n 2^{\frac{N(N-1)}{2} - \frac{n(n-1)}{2}} < \frac{N^n}{n!}2^{\frac{N(N-1)}{2} - \frac{n(n-1)}{2}}. \]

    Последнее выражение меньше $\frac{2^{\frac{N(N-1)}{2}}}{2}$, если $N < (n!)^{\frac{1}{n}} 2^{\frac{n - 1}{2} - \frac{1}{n}}$ (легко получается расписав сравнение).

    Так как $n! > 2^{\frac{n}{2} + 1}$ при $n \geq 3$, то можно взять $N = \left[ 2^{\frac{n}{2}} \right]$.
\end{proof}

\subsubsection*{Нерешенная задача}

Для каких $\lambda$ верна оценка $R(n, n) > \lambda^{(n + o(n))}$?

Наши верхние и нижние оценки: подходит $\lambda = \sqrt{2}$ и не подходит $\lambda = 4$.
\subsection{Числа Рамсея}
\subsubsection*{Обобщение:}

\begin{enumerate}
    \item пары людей(ребра графа знакомств) $\to$ наборы людей по $k$ ($k$-гиперребра).
    
    \item два цвета (знакомы-незнакомы) $\to$ $d$ цветов 
    
    \item будем искать подмножество заранее выбранной мощности, в котором все гиперребра одного цвета.
\end{enumerate}

\begin{defn}
    $N \in \N$ обладает свойством Рамсея $\mathcal{R}(k;m_1,\ldots,m_d)$, если для любой покраски всех $k$-элементных подмножеств $M$, $|M| = N$ в $d$ цветов $\{1,\ldots,d\}$ найдется номер $i$ и подмножество $A \subseteq M$, $|A| = m_i$, такое что все $k$-элементные подмножества множества $A$ покрашены в цвет $i$.  
\end{defn}

\begin{defn}
    Число Рамсея $R(k;m_1, \ldots, m_d)$: наименьшее из натуральных чисел, удовлетворяющих свойству $\mathcal{R}(k;m_1,\ldots,m_d)$.
\end{defn}

\begin{example}
    Утверждение о шести людях: $R(2;3,3) \leq 6$. (на самом деле <<$=$>>).
\end{example}

\begin{theorem}(Рамсея, 1930)
    Для любых натуральных чисел $\{k;m_1,\ldots,m_d\}$ найдется $N \in \N$, обладающее свойством $\mathcal{R}(k;m_1,\ldots,m_d)$. Иными словами, число $R(k;m_1,\ldots,m_d)$ существует и конечно.
\end{theorem}

\begin{proof}
    Отметим, что:
    \begin{enumerate}
        \item $R(1;m_1,\ldots,m_d) = \sum\limits_{i=1}^d m_i - d + 1~~\forall m_1,\ldots,m_d \in \N$.
        
        Действительно, $k=1$ --- покраска элементов множества.

        Отсутствие одноцветного множества $A$ мощности $m_i$ и цвета $i \iff$ в цвет $i$ покрашено не более чем $m_i - 1$ элементов.

        Это возможно для некоторого $i \iff$ всего элементов $\leq \sum\limits_{i=1}^d m_i - d$.
        
        \item Если $\min(m_1,\ldots,m_d) < k$, то $R(k;m_1,\ldots,m_d) = min(m_1,\ldots,m_d)$.
        Так как при $m_i < k$ любое множество мощности $m_i$ нам подойдет в качестве $A$.
    \end{enumerate}
    
    Двойная индукция:
    \begin{itemize}
        \item по $k$ (база $1.$)
        \item по $\sum m_i$ при фиксированном $k$ (база $2.$)
    \end{itemize}
    
    Предположим, что числа Рамсея конечны при меньших значениях $k$ и при данном $k$ при меньшем значении $\sum m_i$. Докажем, что конечно $R(k;m_1,\ldots,m_d)$.

    Если $\min(m_1,\ldots,m_d) < k$, то доказано по $2.$; пусть $min(m_1,\ldots,m_d) \geq k$.

    Обозначим $Q_1 = R(k;m_1 - 1, \ldots, m_d)$; аналагично $Q_i$

    $(i = 2,\ldots,d)$.

    $Q_i$ существуют и конечны по индукционному предположению.

    Положим $N = 1 + R(k-1;Q_1,\ldots,Q_d)$.

    По индукционному предположению по $k$, $N$ существует и конечно.

    Докажем, что $R(k;m_1,\ldots,m_d) \leq N$.

    Рассмотрим $N$-элементное множество $M,~k$-элементные подмножества которого покрашены в цвета от $1$ до $d$.

    Зафиксируем $a \in M$ и покрасим $(k - 1)$-элементные подмножества $M \setminus a = M_1$ в $d$ цветов: всякое $A \subset M_1$ мощности $(k - 1)$ красим в цвет множества $a \cup A$ в $M$.

    Так как $N - 1$ выбрано со свойством $\mathcal{R}(k-1;Q_1,\ldots,Q_d)$, то имеется $B \subseteq M_1,~|B| = Q_i$, такое что все $(k - 1)$-элементные подмножества $B$ имеют цвет $i$. Без ограничения общности $i = 1$. Тогда в $B$ найдется (по определению $Q_i$):
    \begin{itemize}
        \item либо подмножество мощности $m_i$, все $k$-элементные подмножества которого имеют цвет $i$, для некоторого $i \in \{2,\ldots,d\}$;
        \item либо подмножество мощности $m_1 - 1$, все $k$-элементные подмножества которого имеют цвет $1$.
    \end{itemize}
    
    В первом случае сразу имеем то, что нужно. Во втором случае имеем то, что нужно, после добавления к найденному подмножеству мощности $m_1 - 1$ элемента $a$.

    Ч.т.д.
\end{proof}

\begin{notice}
    Обозначим $R(n, m) = R(2;n,m)$.
\end{notice}